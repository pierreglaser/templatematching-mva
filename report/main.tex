\author{Pierre Glaser, Clément Chadebec}
\documentclass{article}
\usepackage{color}
\usepackage{xcolor}
\usepackage{amsmath}
\usepackage{amssymb}
\usepackage{amsthm}
\usepackage{empheq}
\usepackage[most]{tcolorbox}
\usepackage[a4paper, total={6in, 8in}]{geometry}
\usepackage{algorithm}
\usepackage[noend]{algpseudocode}

\makeatletter
\def\BState{\State\hskip-\ALG@thistlm}
\makeatother

\definecolor{materialdarkbg}{RGB}{15,17,26}
\definecolor{materialdarkfg}{RGB}{143, 147, 162}
\definecolor{myblue}{rgb}{.8, .8, 1}
\definecolor{materialblue}{HTML}{82aaff}
\definecolor{materialgray}{HTML}{80869e}


\tcbset{colback=materialgray, colframe=materialblue}

\usepackage{import}
\usepackage{xifthen}
\usepackage{pdfpages}
\usepackage{transparent}

\newcommand{\incfig}[1]{%
    \colorbox{white}{
        \def\svgwidth{\columnwidth}
        \import{./figures/}{#1.pdf_tex}
    }
}

\newtheorem{definition}{Definition}
\newtheorem{proposition}{Proposition}
\newtheorem{theorem}{Theorem}

\title{Rapport de projet - Geodesic Methods and Deformable Models}
\begin{document}
\maketitle
$ \quad $
% \pagebreak
% \part{Questions}


\paragraph{Quel est le problème traité} 
Nous étudierons dans ce rapport l'article \emph{Template Matching via densitives on the
Roto Translation Group}, par Erik J. Bekkers et. Al.

Le problème traité est la \textbf{localisation} d'un objet d'intérêt dans une image par
\textbf{apprentissage supervisé}. Dans les jeux de
données utilisés, l'objet est par exemple le disque optique d'un œil, ou bien la
rétine. Le présupposé est que les problèmes en question bénéficierait d'une prise en
compte des structures d'orientation locale en tout point de l'image.

\paragraph{Quelles sont les équations et les méthodes numériques utilisées} 
\begin{itemize}
    \item La localisation de l'objet d'intérêt se fait par \emph{cross-correlation}: un
        \emph{template} est convolué avec l'image d'intérêt. La valeur maximale du
        résultat est alors la localisation prédite de l'objet. Formellement, si $ f $
        est l'image, et $ x $
    le template:
    \[
        {x}^{\star} = \arg \max_{  } \left ( f \star t \right )(x)
    \] 
    \item Dans le cas standard, $ f $ est définie sur $ \mathbb{R}^2 $. Cependant, $ f $
        peut être transformée (relevée/soulevée?) dans $ \mathbb{R}^3 $, la dimension
        additionnelle décrivant l'état local l'orientation en tout point d'une image.
    \item Le template final est la solution d'un probleme d'optimisation de type
        moindres carrés régularisé, ou régression logistique régularisée.
        Typiquement,
        \[
        t = \arg \min_{  } \sum\limits_{ i=1 }^{ N } \left ( \langle t, p_i \rangle
        - y_i \right )^2 + R(t)
        \] 
        les $ p_i $ sont des templates individuels: 'idéaux': ce sont des images de la
        même taille que $ t $ centrée en le point d'intérêt de l'image $ i $. Si l'on
        suppose que 
        \begin{itemize}
            \item $ \|p_i\| = 1 $
            \item n'importe quelle coupe de $ f_i $ de la taille de $ p_i $ a une norme
                de $ 1 $
        \end{itemize}
        on a $ (p_i \star f_i)(x) = \mathcal  \langle T_x(p_i) f_i[p_i] \rangle  \leq
        \|p_i\| \|f_i[p_i]\| \leq  1$, le maximum étant atteint quand $ f_i $ et $ p_i $ 
        sont alignés, ceci arrivant par construction en $ {x}^{\star}_i $
        ou $ R $ est une pénalité imposant de la régularité à t.
        L'intuition derrière ce problème d'optimisation est.
    \item 
        le template $ t $ doit ensuite être paramétrisé, pour limiter la dimension de
        l'espace
\end{itemize}
\end{document}
