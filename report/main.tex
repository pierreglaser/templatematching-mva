\author{Pierre Glaser, Clement Chadebec}
\documentclass{article}
\usepackage{amsmath}
\usepackage{amssymb}
\usepackage{amsthm}
\usepackage{empheq}
\usepackage[a4paper, total={6in, 8in}]{geometry}


\title{Rapport de projet - Geodesic Methods and Deformable Models}
\begin{document}
\maketitle
$ \quad $
% \pagebreak
% \part{Questions}


\paragraph{Quel est le problème traité} 
hello ok
Nous étudierons dans ce rapport l'article \emph{Template Matching via densitives on the
Roto Translation Group}, par Erik J. Bekkers et. Al.

Le problème traité est la \textbf{localisation} d'un objet d'intérêt dans une image par
\textbf{apprentissage supervisé}. Dans les jeux de
données utilisés, l'objet est par exemple le disque optique d'un œil, ou bien la
rétine. Le présupposé est que les problèmes en question bénéficierait d'une prise en
compte des structures d'orientation locale en tout point de l'image.

\paragraph{Quelles sont les équations et les méthodes numériques utilisées} 
\begin{itemize}
    \item La localisation de l'objet d'intérêt se fait par \emph{cross-correlation}: un
        \emph{template} est convolué avec l'image d'intérêt. La valeur maximale du
        résultat est alors la localisation prédite de l'objet. Formellement, si $ f $
        est l'image, et $ x $
    le template:
    \[
        {x}^{\star} = \arg \max_{  } \left ( f \star t \right )(x)
    \] 
    \item Dans le cas standard, $ f $ est définie sur $ \mathbb{R}^2 $. Cependant, $ f $
        peut être transformée (relevée/soulevée?) dans $ \mathbb{R}^3 $, la dimension
        additionnelle décrivant l'état local l'orientation en tout point d'une image.
    \item Le template final est la solution d'un probleme d'optimisation de type
        moindres carrés régularisé, ou régression logistique régularisée.
        Typiquement,
        \[
        t = \arg \min_{  } \sum\limits_{ i=1 }^{ N } \left ( \langle t, p_i \rangle
        - y_i \right )^2 + R(t)
        \] 
        les $ p_i $ sont des templates individuels: ``idéaux'': ce sont des patches,
        extraits des images, de la même taille que $ t $ centrée en le point d'intérêt
        de l'image $ i $. Si l'on suppose que 
        \begin{itemize}
            \item $ \|p_i\| = 1 $
            \item n'importe quelle coupe de $ f_i $ de la taille de $ p_i $ a une norme
                de $ 1 $
        \end{itemize}
        on a $ (p_i \star f_i)(x) = \mathcal  \langle T_x(p_i) f_i[p_i] \rangle  \leq
        \|p_i\| \|f_i[p_i]\| \leq  1$, le maximum étant atteint quand $ f_i $ et $ p_i $ 
        sont alignés, ceci arrivant par construction en $ {x}^{\star}_i $
        ou $ R $ est une pénalité imposant de la régularité à t.
    \item A ce moment la, le template est une variable dans $ \mathbb{R}^N$, N étant le
        nombre de pixels du patch. On pourrait alors optimiser chaque pixel
        respectivement, mais il serait alors difficile d'imposer des contraintes de
        régularité du patch final. Une autre maniere de faire est de parametriser le
        patch comme une combinaison linéaire de fonctions régulière: c'est l'approche
        suivie dans cet article, qui utilise come fonction les B-splines. Le tempate
        s'ecrit alors
        \[
            \sum\limits_{ k=1 }^{ N_k } \sum\limits_{ l=1 }^{ N_l } c_k B^n \left (
            \frac{x}{s_k} - k \right ) B^n \left ( \frac{y}{s_{l}} - l \right )
        \] 
\end{itemize}
somefunction

\paragraph{Pouvez vous situer l'article par rapport aux méthodes étudiées en cours et le
comparer à des sujets proches évoqués en cours} 
La différence principale avec les sujets étudiés en cours est que cette méthode est une
méthode d'apprentissage supervisée et automatique alors que les méthodes étudiées
pendant le cours sont non-supervisées et demandent généralement une intervention humaine
au moment de l'initialisation.\newline
Une caractéristique commune ce ces deux méthodes est la présence d'un terme de
pénalisation sur la régularité de la solution au problème d'optimisation, les deux
faisant intervenir le gradient le la paremétrisation.
\end{document}
