\author{Pierre Glaser, Clement Chadebec}
\documentclass{article}
\usepackage{amsmath}
\usepackage{amssymb}
\usepackage{amsthm}
\usepackage{empheq}
\usepackage[most]{tcolorbox}
\usepackage[a4paper, total={6in, 8in}]{geometry}
\usepackage{algorithm}
\usepackage[noend]{algpseudocode}
\usepackage[francais]{babel}
\usepackage[utf8]{inputenc}
\usepackage[T1]{fontenc}



\makeatletter
\def\BState{\State\hskip-\ALG@thistlm}
\makeatother

\definecolor{materialdarkbg}{RGB}{15,17,26}
\definecolor{materialdarkfg}{RGB}{143, 147, 162}
\definecolor{myblue}{rgb}{.8, .8, 1}
\definecolor{materialblue}{HTML}{82aaff}
\definecolor{materialgray}{HTML}{80869e}


\tcbset{colback=materialgray, colframe=materialblue}

\color{materialdarkfg}
\pagecolor{materialdarkbg}
\usepackage{import}
\usepackage{xifthen}
\usepackage{pdfpages}
\usepackage{transparent}

\newcommand{\incfig}[1]{%
    \colorbox{white}{
        \def\svgwidth{\columnwidth}
        \import{./figures/}{#1.pdf_tex}
    }
}

\newtheorem{definition}{Definition}
\newtheorem{proposition}{Proposition}
\newtheorem{theorem}{Theorem}

\title{Rapport de projet - Geodesic Methods and Deformable Models}
\begin{document}
\maketitle
$ \quad $
% \pagebreak
% \part{Questions}


\paragraph{Quel est le problème traité} 
Nous étudierons dans ce rapport l'article \emph{Template Matching via densitives on the
Roto Translation Group}, par Erik J. Bekkers et. Al.

Le problème traité est la \textbf{localisation} d'un objet d'intérêt dans une image par
\textbf{apprentissage supervisé}. Dans les jeux de
données utilisés, l'objet est par exemple le disque optique d'un œil, ou bien la
rétine. Le présupposé est que les problèmes en question bénéficieraient d'une prise en
compte des structures d'orientation locale en tout point de l'image.

\paragraph{Quelles sont les équations et les méthodes numériques utilisées} 
\begin{itemize}
    \item La localisation de l'objet d'intérêt se fait par \emph{cross-correlation}: un
        \emph{template} est convolué avec l'image d'intérêt. La valeur maximale du
        résultat est alors la localisation prédite de l'objet. Formellement, si $ f $
        est l'image, et $ x $
    le template:
    \[
        {x}^{\star} = \arg \max_{  } \left ( f \star t \right )(x)
    \] 
    \item Dans le cas standard, $ f $ est définie sur $ \mathbb{R}^2 $. Cependant, $ f $
        peut être transformée (relevée/soulevée?) dans $ \mathbb{R}^3 $, la dimension
        additionnelle décrivant l'état local et l'orientation en tout point d'une image.
    \item Le template final est la solution d'un probleme d'optimisation de type
        moindres carrés régularisés, ou régression logistique régularisée.
        Typiquement,
        \[
        t = \arg \min_{  } \sum\limits_{ i=1 }^{ N } \left ( \langle t, p_i \rangle
        - y_i \right )^2 + R(t)
        \] 
        les $ p_i $ sont des templates individuels: ``idéaux'': ce sont des patches,
        extraits des images, de la même taille que $ t $ centrée en le point d'intérêt
        de l'image $ i $. Si l'on suppose que 
        \begin{itemize}
            \item $ \|p_i\| = 1 $
            \item n'importe quelle coupe de $ f_i $, une image, de la taille de $ p_i $ a une norme
                de $ 1 $
        \end{itemize}
        on a $ (p_i \star f_i)(x) = \mathcal  \langle T_x(p_i) f_i[p_i] \rangle  \leq
        \|p_i\| \|f_i[p_i]\| \leq  1$, le maximum étant atteint quand $ f_i $ et $ p_i $ 
        sont alignés, ceci arrivant par construction en $ {x}^{\star}_i $
        ou $ R $ est une pénalité imposant de la régularité à t.
    \item A ce moment là, le template est une variable dans $ \mathbb{R}^N$, N étant le
        nombre de pixels du patch. On pourrait alors optimiser chaque pixel
        respectivement, mais il serait alors difficile d'imposer des contraintes de
        régularité du patch final. Une autre manière de faire est de paramétriser le
        patch comme une combinaison linéaire de fonctions régulière: c'est l'approche
        suivie dans cet article, qui utilise come fonction les B-splines. Le template
        s'ecrit alors
        \[
            t(x, y) = \sum\limits_{ k=1 }^{ N_k } \sum\limits_{ l=1 }^{ N_l } c_k B^n \left (
            \frac{x}{s_k} - k \right ) B^n \left ( \frac{y}{s_{l}} - l \right )
        \] 
        Une expression qui admet alors un gradient, ceci permettant d'ajouter au
        probleme d'optimisation un terme de la forme
        \[
            \int\limits_{  }^{  } \| \nabla_{  } t(x, y) \|^2 dx dy
        \] 
\end{itemize}

\paragraph{Pouvez vous situer l'article par rapport aux méthodes étudiées en cours et le
comparer à des sujets proches évoqués en cours} 
La différence principale avec les sujets étudiés en cours est que cette méthode est une
méthode d'apprentissage supervisée et automatique alors que les méthodes étudiées
pendant le cours sont non-supervisées et demandent généralement une intervention humaine
au moment de l'initialisation.\newline
Une caractéristique commune de ces deux méthodes est la présence d'un terme de
pénalisation sur la régularité de la solution du problème d'optimisation, les deux
faisant intervenir le gradient le la paremétrisation: on rapelle que le problème des
contours actifs s'écrit:
\[
    \int\limits_{  \Omega }^{  } \left ( w_1^2 \|C'(s)\|^2 + w_2^2 \|C''(s)\|^2 +
    P(C(s)) \right )ds
\] 
alors que le problème étudié concerne la résolution de 
\[
    t = \arg \min_{  } \sum\limits_{ i=1 }^{ N } \left ( \langle t, p_i \rangle
    - y_i \right )^2 + \lambda \int\limits_{  }^{  } \| \nabla_{  } t(x, y) \|^2 dx dy + \mu \| t \|^2
\]
\paragraph{Quelle est l'originalité du travail (selon les auteurs).} 
Ce travail s'inscrit dans la continuité d'une  série d'article des auteurs: la majorité
des concepts utilisés dans cet articles on été introduits dans des publications
précédentes:
\begin{itemize}
    \item l'utilisation des scores d'orientation a été introduit en [32]
    \item l'utilisation des cake wavelets a été introduit dans la thèse de l'auteur.
    \item la cross correlation est une technique bien connue pour détecter des points
        d'intérêts.
\end{itemize}
Les auteurs proposent alors une méthode de résolution pour une reformulation de la 
``regression logistique'' permettant à cette méthode d'atteindre l'état de l'art en
termes de performances, ainsi qu'une série d'experiences très completes
comparant l'influence des différents types de régularisation et l'importance des
orientation scores. Enfin, est donnée en appendice une interprétation mathématique d'une
version simplifiée du problème régularisé dans $ SE2(\mathbb{R}) $
\paragraph{Quels sont les résultats nouveaux importants qui en découlent.}
Le résultat le plus important ici est à nos yeux la preuve expérimentale que cette
méthode atteint l'état de l'art pour les jeux de données considérées.
\paragraph{Voyez-vous des faiblesses dans l'approche présentée et avez-vous des idées
pour y faire face?}

\paragraph{Expériences}

Afin de mieux comprendre leur fonctionnement et de pouvoir les comparer dans 
divers cadres d'études, nous avons recodé les différentes méthodes présentées dans cet article 
en langage \textit{python}. Le code complet peut être trouvé au lien suivant ???????
Pour mémoire, 5 types de construction de templates sont envisagés:
\begin{itemize}
    \item A: Template obtenu par moyenne de tous les patches positifs (i.e. centrés sur la zone d'intérêt) et normalisation.
    \item B: Template optimisé sans régularisation ($\mu =0$, $\lambda = 0$)
    \item C: Template optimisé avec $\mu$ ($\lambda=0$)
    \item D: Template optimisé avec $\lambda$ ($\mu = 0$)
    \item E: Template optimisé avec $\mu$ et $\lambda$
\end{itemize}

Pour chacune de ses constructions, un module de regression linéaire et regression logistique est employé dans 
les espaces $\mathbb{R}^2$ et $SE(2)$. Afin de réaliser les expérience dont les résultats sont présentés plus 
ont été réalisée en considérant un ensemble de 500 images dont 80\% (400 images) constitue l'ensemble d'entrainement 
des modèles et les 20 \% (100 images) restants l'ensemble de test. Sur chacune des images d'entrainement, un patch 
positif de taille $N_x \times N_y = 101 \times 101$ centré autour des coordonnées de l'oeil gauche et un patch négatif 
de même de taille tiré aléatoirement en dehors des zones d'intérêts (oeil droit et gauche) sont crées. Un détection est 
ensuite considérée "bonne" sur le point de détection se situe dans rayon de $10$ pixels autour de la position réelle de l'oeil 
gauche.



\paragraph{Template Matching dans $\mathbb{R}^2$}
Premièrement, une comparaison des résultats obtenus avec les différentes methodes a été réalisée dans $\mathbb{R}^2$.
Les résultats sont reportés dans \ref{table: R2}. Afin de trouver les valeurs optimales de $\mu$ (resp $\lambda$) dans les expériences C (resp. D), nous avons décidé 
de procèder de façon empirique en faisant varier les valeurs de ces deux paramètres entre $10^{-4}$ et $1$. Afin de trouver le mailleur couple pour l'expérience E, nous avons 
choisi de tout d'abord fixer la valeur de $\mu$ à la valeur optimale trouvée lors de l'expérience C, puis de faire varier $\lambda$ de  $10^{-4}$ et $1$.

\begin{table}[h!]
    \centering
    \begin{tabular}{|c|c|c|}
        \hline
        Methode & Score (\%)\\
        \hline
        \hline
        $A^{\mathbb{R}^2}$& 61.0 \% \\
        \hline
        $B_{\text{lin}}^{\mathbb{R}^2}$& 3.0 \%    \\
        $C_{\text{lin}}^{\mathbb{R}^2}$& 62.0 \%   \\
        $D_{\text{lin}}^{\mathbb{R}^2}$& 38.0 \%   \\
        $E_{\text{lin}}^{\mathbb{R}^2}$& 62.0 \%   \\
        \hline
        $B_{\text{log}}^{\mathbb{R}^2} $& 0.0 \%   \\ 
        $C_{\text{log}}^{\mathbb{R}^2} $& 63.0 \%  \\ 
        $D_{\text{log}}^{\mathbb{R}^2} $& 7.0 \%   \\ 
        $E_{\text{log}}^{\mathbb{R}^2} $& 63.0 \%  \\ 
        \hline
    \end{tabular}
    \caption{Résultats obtenus template matching dans $\mathbb{R}^2$}
    \label{table: R2}
\end{table}

La table \ref{table: R2} montre qu'un template relativement simple ne correspondant finlement qu'à la moyenne de tous les patches positifs démontre tout de même de bons
résultats comparé à des modèles de régression plus évolués. Effectivement, si non complés à de mecanismes de régularisation, les modèles $B_{\text{log}}^{\mathbb{R}^2} $ et 
$B_{\text{lin}}^{\mathbb{R}^2} $ ne demontre que de faibles performances. Par ailleurs, un bon choix de paramètres de régularisation peut conduire à améliorer drastiquement les 
performances des regressions dont le meilleur résulat est observé pour $\mu = 10^{-1}$ et $\lambda = 10^{-2}$ pour la regression linéaire et $\mu = 10^{-1}$ et $\lambda = 10^{-3}$ pour la regression logistique.


\paragraph{Template Matching dans $SE(2)$}
Les mêmes expériences ont ensuite été reproduites en utilisant l'espace $SE(2)$ et donc ajoutant également l'orientation score dans les régressions.
Cependant, dans cette section les paramètres le paramètres $\mu \in [10^{-4}, 1]$ et $\lambda \in [10^{-4}, 100]$. En ce qui concernent les paramètres inhérents à la matrices de 
régularisation ont été fixé comme suit $D_{\xi \xi}=1, D_{\eta \eta} = 0$ et $D_{\theta \theta} = 10^{-2}$. Nous n'avons pas fait varié ces paramètres qui ont été choisis en accord avec les auteurs 
de l'artcile présenté. Les resultats peuvent être observés en \ref{table: SE(2)}.

\begin{table}[h!]
    \centering
    \begin{tabular}{|c|c|c|}
        \hline
        Methode & Score (\%)\\
        \hline
        \hline
        $A^{SE(2)}$&  64.0 \% \\
        \hline
        $B_{\text{lin}}^{SE(2)}$&     3.0  \%   \\
        $C_{\text{lin}}^{SE(2)}$&     68.0 \%   \\
        $D_{\text{lin}}^{SE(2)}$&     61.0 \%   \\
        $E_{\text{lin}}^{SE(2)}$&     68.0 \%   \\
        \hline
        $B_{\text{log}}^{SE(2)} $&    \%   \\ 
        $C_{\text{log}}^{SE(2)} $&    \%   \\ 
        $D_{\text{log}}^{SE(2)} $&    \%   \\ 
        $E_{\text{log}}^{SE(2)} $&    \%   \\ 
        \hline
    \end{tabular}
    \caption{Résultats obtenus template matching dans $SE(2)$}
    \label{table: SE(2)}
\end{table}

\end{document}

